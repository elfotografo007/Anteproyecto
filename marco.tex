\chapter{Marco Referencial}
\label{sec:marco}

\section{Marco de Antecedentes}
Entre las herramientas que proporciona Foursquare para atraer y recompensar clientes son las siguientes:
\begin{itemize}
\item Dejarse descubrir, la cual permite anunciar la localizaci\'on de un comercio y facilitar que las dem\'as personas lo encuentren.
\item Atraer clientes, la cual permite atraer clientes nuevos o recompensar clientes fieles. Esto se puede hacer mediante actualizaciones o especiales.
\item Monitorear tr\'afico, que es usar las anal\'iticas gratis para aprender m\'as acerca de los clientes.
\end{itemize}
Entre las herramientas anteriores, la de monitoreo de tr\'afico permite saber cu\'antas personas circulan cada d\'ia por la tienda, adem\'as que brinda estad\'isticas en tiempo de real acerca de  las actualizaciones y especiales que est\'an activos en un comercio espec\'ifico.
\paragraph{}
Una herramienta adicional que brinda Foursquare, es contactarlos a ellos para hacer campa\~nas de mercadeo, bajo un esquema de pago. Esto todav\'ia es un programa piloto, los cuales han sido utilizados por empresas como The New York Times, Forbes, y WSJ Live.\footnote{Foursquare. Advertise with Foursquare[Online]. Disponible en internet \textless http://business.foursquare.com/advertise/\textgreater}
\paragraph{}
HootSuite es una herramienta de terceros para administrar Foursquare y dem\'as redes sociales. Este sitio es uno de los m\'as usados en el mundo para este tipo de tareas y entre sus clientes figuran Mc Donald’s, Pepsico, Sony Music, Lamborghini, Virgin, entre otros. Lo que ofrece Hootsuite para Foursquare es lo siguiente:
\paragraph{}
''La gesti\'on de Foursquare con HootSuite combina marketing social, servicio de localizaci\'on y juegos que animan a los m\'as populares a hacer \textit{''check in''} en lugares y cont\'arselo a sus amigos, compartir consejos, e incluso a convertirse en el ''alcalde'' de un lugar.''\footnote{Hoot Suite. Gesti\'on de redes sociales[Online]. Disponible en internet \textless http://hootsuite.com/features/social-networks\textgreater}

\section{Marco Te\'orico}
\subsection{\textit{Big Data}}\footnote{ZIKOPOULOS PAUL et al. Understanding Big Data: Analytics for Enterprise Class Hadoop and Streaming Data. McGraw-Hill. 2012. pp. 3-31}
\textit{Big Data} es un concepto muy de moda en los \'ultimos d\'ias. El t\'ermino hace referencia a la consulta, almacenamiento, recopilaci\'on, procesamiento, an\'alisis y administraci\'on de vol\'umenes exageradamente grandes de datos. Tambi\'en engloba la extracci\'on de informaci\'on valiosa y significativa de estas grandes cantidades de datos, al igual que proveer inteligencia y “conocimiento” a ra\'iz de los mismos. La definici\'on exacta de “vol\'umenes exageradamente grandes de datos” var\'ia en funci\'on del autor y la organizaci\'on o empresa que trabajo con estos datos. Para algunos \textit{Big Data} significa vol\'umenes superiores a 1TB, mientras que para otros es por el orden de los PB.
\paragraph{}
Generalmente, el enfoque de \textit{Big Data} es el an\'alisis de datos no estructurados, debido a que este tipo de datos son los que actualmente tienen el crecimiento m\'as r\'apido. Algunos de estos datos no estructurados pueden ser im\'agenes, datos de sensores, telemetr\'ia, videos, documentos, entre otros. Esto datos no tienen un modelo de datos estructurado predefinido, o simplemente no hay formas de adaptarlos bien a modelos estructurados o relacionales. Adicionalmente \textit{Big Data} intenta disminuir el tiempo que transcurre entre la adquisici\'on de los datos, y la aplicaci\'on del conocimiento o las decisiones basados en dichos datos, en especial en las \'areas de inteligencia de negocios. Tambi\'en busca brindar estas soluciones con costos relativamente bajos y que sean altamente escalables.
\paragraph{}
Normalmente en el \textit{Data Warehouse} tradicional, los procesos y las depuraciones que sufren los datos son muy rigurosas, debido a la concepci\'on que los datos de Data Warehouse deben ser de una calidad elevada, ya que, los datos almacenados en \textit{Data Warehouse} se presumen son de alt\'isimo valor comercial. \textit{Big Data} toma un enfoque diferente, y asume un bajo valor comercial de los datos hasta que se demuestre lo contrario. Los costos de llevar todos los datos capturados por la empresa a \textit{Data Warehouse} son demasiado altos, mientras que en \textit{Big Data} son relativamente bajos. No se quiere indicar que \textit{Big Data} es mejor que el \textit{Data Warehouse} tradicional, ni que es un reemplazo (aunque podr\'ia serlo), se quiere hacer entender que pueden ser vistos
como complementarios.
\paragraph{}
Hay tres caracter\'isticas que definen \textit{Big Data}: volumen, variedad y velocidad. 
\paragraph{}
Las empresas est\'an enfrent\'andose a vol\'umenes masivos de datos, los cuales las est\'an abrumando. Adicionalmente, a nivel que la cantidad de datos disponibles para an\'alisis aumenta, el porcentaje de datos que se pueden procesar y analizar est\'a disminuyendo.
\paragraph{}
La variedad de los datos se refiere a que en este momento solamente el 20\% de los datos del mundo son estructurados. Ahora debemos almacenar y procesar datos estructurados, semi-estructurados, no estructurados, brutos, relacionales, entre otros. En este momento, para obtener una ventaja, las empresas deben analizar todos los tipos de datos, tanto relacionales como no relacionales: texto, datos de sensores, audio, video, transacciones, etc.
\paragraph{}
La velocidad, m\'as que la tasa de entrada y generaci\'on de los datos, se refiere a que muchos datos deben ser procesados en tiempo real, y que ganar segundos, microsegundos o milisegundos en cuanto a su procesamiento puede significar una ventaja competitiva mayor. Al contrario del procesamiento tradicional de consultas donde se hace una consulta a un dato hist\'orico y se trabaja, \textit{Big Data} busca adicionalmente hacer una consulta continua, para obtener resultados actualizados continuamente.
\paragraph{}
El problema del \textit{Big Data} se ataca de dos formas: almacenamiento y procesamiento. Seg\'un los documentos, el problema del almacenamiento est\'a en su mayor\'ia resuelto, y se cuentan con distintos sistemas de archivos que cumplen esta tarea. El problema del procesamiento y an\'alisis todav\'ia tiene mucho campo por delante para investigar.
\subsection{\textit{Map Reduce}}\footnote{USENIX Association Berkeley [Online]. CA, USA:  Dean Jeffrey y  Ghemawat Sanjay. MapReduce: simplified data processing on large clusters. 2004 Disponible en internet \textless http://dl.acm.org/citation.cfm?id=1251264\textgreater}
En el campo del an\'alisis de datos no estructurados, hay un \textit{framework} muy importante llamado \textit{MapReduce}, publicado por Google en 2004. \textit{MapReduce} se corre en dos fases: Mapeo y reducci\'on. Estas fases son funciones que reciben como entrada pares llave/valor, los procesan y entregan como resultado otro conjunto llave/valor.
\paragraph{}
El paradigma MapReduce, simplifica las operaciones al abstraer toda la operaci\'on de procesamiento a un conjunto de acciones \textit{“Map”} y \textit{“Reduce”} simples, encadenadas para producir un resultado. Lo m\'as atractivo de esto, es que el \textit{framework} que se utiliza para trabajar con MapReduce se encargar\'a de manejar toda la complejidad relacionada con la distribuci\'on de tareas, paralelizaci\'on, balanceo de carga, tolerancia a fallos del proceso.
\paragraph{}
Un proyecto que se considera importante en el campo de \textit{Big Data} es Apache Hadoop, el cual es un proyecto de c\'odigo abierto que implementa los conceptos de Google sobre \textit{MapReduce}. Hadoop apunta a implementar un \textit{cluster} grande de nodos de bajo costo, que sea f\'acilmente escalable y pueda proveer cierto grado de tolerancia a fallos de los nodos.
\paragraph{}
Por el lado de la infraestructura de red, es necesario contar con una red que nos provea la redundancia requerida y que se pueda escalar a medida que el \textit{cluster} crece. Las conexiones de red son usadas en forma intensiva en las operaciones de lectura y escritura a los sistemas de archivos y en las distintas fases de los algoritmos de procesamiento y an\'alisis. La naturaleza del tr\'afico que circula en los \textit{cluster} de \textit{Big Data} es generalmente explosiva, por lo tanto es importante que la red pueda manejar esto y que adicionalmente tenga una buena profundidad de cola(\textit{buffer}), para evitar p\'erdidas de paquetes.
\subsection{Motivaci\'on del consumidor}
\begin{quote}
Las necesidades humanas – necesidades del consumidor – son el fundamento de todo el marketing moderno. Las necesidades constituyen la esencia del concepto marketing. La clave de la supervivencia, la rentabilidad y el crecimiento de una compañ\'ia en un mercado altamente competitivo es su capacidad para identificar y satisfacer necesidades insatisfechas del consumidor mejor y m\'as r\'apido que la competencia.
Los mercad\'ologos no crean las necesidades, aunque en algunos casos pueden hacer que los consumidores est\'en m\'as conscientes de necesidades que anteriormente no hab\'ian sentido.\footnote{SCHIFFMAN , Leon G. KANUK , Leslie L. El comportamiento del consumidor. Pearson Educacion (Agosto 2005). p. 85}
\end{quote}
\subsection{Motivaci\'on como una fuerza psicol\'ogica}
\begin{quote}
La motivaci\'on se define como la fuerza impulsora dentro de los individuos que los empuja a la acci\'on. Esta fuerza impulsora se genera por un estado de tensi\'on que existe como resultado de una necesidad insatisfecha. Los individuos se esfuerzan tanto consciente como subconscientemente por reducir dicha tensi\'on mediante un comportamiento que, seg\'un sus expectativas, satisfacer\'a sus necesidades y, de esa manera, mitigara el estr\'es que padecen. Las metas espec\'ificas que eligen y los patrones que realizan para alcanzar sus metas son resultado del pensamiento y el aprendizaje individual.\footnote{IBID. p. 87}
\end{quote}
\section{Marco Conceptual}
\subsection{Georreferenciaci\'on}\footnote{King, Kevin F., Geolocation and Federalism on the Internet: Cutting Internet Gambling’s Gordian Knot (July 14, 2009). Columbia Science and Technology Law Review, Vol. XI, 2010. p. 58}
Por georreferenciaci\'on o geolocalizaci\'on se entiende la posici\'on de un objeto en un sistema de coordenadas geogr\'aficas, que identifican la localizaci\'on f\'isica de un usuario final de internet.
\subsection{Fases \textit{Map} y \textit{Reduce}}
En la fase de mapeo, se divide la carga de trabajo en cargas de trabajo m\'as peque\~nas, luego las procesa y finalmente las pasa a la siguiente fase. En la fase de reducci\'on se analiza y se mezcla la entrada de la fase anterior para producir la salida final del sistema. Esta salida es escrita en el sistema de archivos del cl\'uster.
\subsection{Foursquare}
\begin{quote}
Foursquare es una aplicaci\'on gratuita que te ayuda a ti y a tus amigos a sacar el m\'aximo provecho de d\'onde est\'an.​ Cada vez que salgas,​ usa Foursquare para compartir y guardar los lugares que visitas.​ Y cuando necesites ideas sobre qu\'e hacer despu\'es,​ te daremos recomendaciones personalizadas y ofertas seg\'un donde hayan estado t\'u,​ tus amigos y la gente con tus mismos gustos.​
\paragraph{}
Ya sea que est\'es por viajar por el mundo, organizando una noche con amigos o tratando de elegir el mejor plato en tu restaurante local, Foursquare es el compa\~nero ideal.\footnote{Foursquare. About foursquare[Online]. Disponible en internet \textless https://foursquare.com/about/\textgreater}
\end{quote}
\subsubsection{Algunos conceptos claves de Foursquare}
\begin{description}
\item[\textit{Check-ins}:] Las personas usan Foursquare para encontrar nuevos lugares para ir, e ingresar (\textit{“check in”}) para decirle a sus amigos d\'onde est\'an y qu\'e es grandioso. Foursquare tambi\'en utiliza \textit{check-ins} pasados para personalizar las recomendaciones de las personas.
\item[\textit{Tips}:] Las personas dejan breves recomendaciones en lugares para sus amigos y otros para descubrir, como qu\'e plato ordenar, el mejor tiempo para visitar o c\'omo encontrar el cuarto trasero secreto.
\item[Listas:] Las personas hacen listas de lugares a donde quieren ir, o listas de lugares que disfrutan, como los mejores puntos para bailar o sus restaurantes vegetarianos favoritos.
\item[Especiales:] Los especiales son recompensas que las personas pueden obtener por hacer \textit{check-in} en un negocio o evento.
\item[Actualizaciones:] Las actualizaciones son mensajes que se pueden enviar desde las tiendas hacia los clientes
\end{description}
\subsection{Mercado objetivo}
\begin{quote}
El mercado objetivo hace alusi\'on al espacio del mercado que tienen en com\'un la oferta y la demanda. Comprende entre sus elementos m\'as importantes el alcance geogr\'afico, los canales de distribuci\'on, las categor\'ias de productos comerciados, el repertorio de competidores directos e indirectos, los t\'erminos de intercambio, y a los representantes de la demanda entre los que se encuentran influenciadores y prospectos compradores.\footnote{KOTLER, Philip.  Fundamentos de Marketing (6ª edici\'on). Pearson Educaci\'on de M\'exico. 2003. p. 123}
\end{quote}
\pagebreak